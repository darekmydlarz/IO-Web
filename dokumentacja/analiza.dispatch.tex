\chapter{Analiza pliku wynikowego Dispatch Ridera}
Program Dispatch Rider generuje przy odpowiedniej konfiguracji (recording=''true'') plik wyjściowy w formacie XML, który przedstawia model przetwarzanego problemu transportowego. W założeniu nasza aplikacja webowa ma otrzymywać taki plik wynikowy, otwierać go, a następnie parsować w celu odczytania istotnych dla wizualizacji danych. Dane te będą następnie przetwarzane do formatu akceptowanego przez wybraną przez nas bibliotekę graficzną (JavaScript InfoVis Toolkit).

\section{Struktura pliku wynikowego Dispatch Ridera}
Plik generowany przez Dispatch Ridera ma następującą strukturę:
\begin{verbatim}
<simulation_measures>
    <measures comId="42" timestamp="0">
        <holon id="0">
            <measure name="NumberOfCommissions">1.0</measure>
            <measure name="AverageDistanceFromCurLocationToBaseForAllCommissions">
            	83.2608973717183</measure>
            <measure name="AverageDistPerCommissionBeforeChange">
           		34.85191972054854</measure>
        </holon>
    </measures>
    <measures comId="51" timestamp="0">
        <holon id="0">
            <measure name="NumberOfCommissions">1.0</measure>
            <measure name="AverageDistanceFromCurLocationToBaseForAllCommissions">
            	82.30488721735766</measure>
            <measure name="AverageDistPerCommissionBeforeChange">
            	34.85191972054854</measure>
        </holon>
        <holon id="1">
            <measure name="NumberOfCommissions">1.0</measure>
            <measure name="AverageDistanceFromCurLocationToBaseForAllCommissions">
            	66.91504977153873</measure>
            <measure name="AverageDistPerCommissionBeforeChange">
            	21.8026373564547</measure>
        </holon>
    </measures>
</simulation_measures>
\end{verbatim}
Dzięki temu, że jest to plik zgodny ze specyfikacją XML, do parsowania użyjemy JavaScriptu. Uściślając skorzystamy z inwentarza biblioteki jQuery i być może pluginu do przekształcania formatu XML w format JSON (przykładowy sposób zwykłego parsingu przy jej użyciu: \url{http://think2loud.com/224-reading-xml-with-jquery/}, \url{http://www.switchonthecode.com/tutorials/xml-parsing-with-jquery} oraz pluginu do konwersji pomiędzy XML i JSON: \url{http://www.fyneworks.com/jquery/xml-to-json/}).
\section{Struktura pliku dla biblioteki graficznej}
Powyższy plik XML będziemy przekształcać do formatu JSON i postaci podobnej do poniższej:
\begin{verbatim}
var json = [  
    {  
      "adjacencies": [  
          "graphnode21",   
          {  
            "nodeTo": "graphnode1",  
            "nodeFrom": "graphnode0",  
            "data": {  
              "$color": "#557EAA"  
            }  
          }, {  
            "nodeTo": "graphnode13",  
            "nodeFrom": "graphnode0",  
            "data": {  
              "$color": "#909291"  
            }  
          }  
      ],  
      "data": {  
        "$color": "#83548B",  
        "$type": "circle",  
        "$dim": 10  
      },  
      "id": "graphnode0",  
      "name": "graphnode0"  
    }, {  
      "adjacencies": [  
          {  
            "nodeTo": "graphnode2",  
            "nodeFrom": "graphnode1",  
            "data": {  
              "$color": "#557EAA"  
            }  
          }, {  
            "nodeTo": "graphnode4",  
            "nodeFrom": "graphnode1",  
            "data": {  
              "$color": "#909291"  
            }  
          }
      ],  
      "data": {  
        "$color": "#EBB056",  
        "$type": "circle",  
        "$dim": 11  
      },  
      "id": "graphnode1",  
      "name": "graphnode1"  
    }
];
\end{verbatim}