\chapter{Oprogramowanie schedulera}
W celu obsługi kolejnych zadań przesyłanych do systemu, powstał szkielet oprogramowania zarządzającego kolejkowaniem oraz uruchamianiem Dispatch Ridera z kolejnymi zestawami danych.


Skrypt przeszukuje podany w parametrach startowych katalog systemu plików, poszukując nowo dodanych zestawów danych. W przypadku gdy taki zestaw się pojawia, umieszcza plik 'configuration.xml' zestawu w kolejce zadań do wykonania.


Następnie scheduler sprawdza czy pojawiły się pliki wynikowe działania programu, co jest jednoznaczne z zakonczeniem wykonywania danego zadania. Jeśli pliki znajdują się w katalogu wynikowym, wówczas uruchamiane jest kolejne zadanie z kolejki. 


Kwestią istotną jest fakt, iż aby system zadziałał, plik 'configuration.xml' musi znajdować się w katalogu głównym Dispatch Ridera. Ów wymóg jest podyktowany zapewne założeniem projektowym lub błędem DR. W związku z tym, plik ten dla każdego nowo uruchamianego zadania jest kopiowany do głównego katalogu przed uruchomienim procesu. 


Kwestią pozostałą do uzgodnienia pozostaje w tej chwili struktura katalogów tworzona podczas ładowania plików z danymi do systemu - scheduler działa poprawnie, jednak konieczne jest wypracowanie polityki zarządzania danymi wejściowymi. Dyskusję dotyczącą tego zagadnienia przeprowadzimy podczas tworzenia systemu wczytywania i parsowania danych wejściowych.