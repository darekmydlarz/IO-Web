\lstset{language=Java}
\chapter{Oprogramowanie schedulera}
W celu obsługi kolejnych zadań przesyłanych do systemu, powstał szkielet oprogramowania zarządzającego kolejkowaniem oraz uruchamianiem Dispatch Ridera z kolejnymi zestawami danych.


Skrypt przeszukuje podany w parametrach startowych katalog systemu plików, poszukując nowo dodanych zestawów danych. W przypadku gdy taki zestaw się pojawia, umieszcza plik 'configuration.xml' zestawu w kolejce zadań do wykonania.


Następnie scheduler sprawdza czy pojawiły się pliki wynikowe działania programu, co jest jednoznaczne z zakonczeniem wykonywania danego zadania. Jeśli pliki znajdują się w katalogu wynikowym, wówczas uruchamiane jest kolejne zadanie z kolejki. 


Kwestią istotną jest fakt, iż aby system zadziałał, plik 'configuration.xml' musi znajdować się w katalogu głównym Dispatch Ridera. Ów wymóg jest podyktowany zapewne założeniem projektowym lub błędem DR. W związku z tym, plik ten dla każdego nowo uruchamianego zadania jest kopiowany do głównego katalogu przed uruchomienim procesu. 


Kwestią pozostałą do uzgodnienia pozostaje w tej chwili struktura katalogów tworzona podczas ładowania plików z danymi do systemu - scheduler działa poprawnie, jednak konieczne jest wypracowanie polityki zarządzania danymi wejściowymi. Dyskusję dotyczącą tego zagadnienia przeprowadzimy podczas tworzenia systemu wczytywania i parsowania danych wejściowych.


Dodane 25.05.2012

\textbf{Problem z uzyskaniem wyników działania programu}
Jak pisaliśmy powyżej, do uruchomienia Dispatch Ridera wykorzystujemy napisany w jeżyku Java skrypt. Wykonywane jest następujące polecenie systemowe:

\begin{lstlisting}
command = "java -cp " + mainPath + "/jar/DTP.jar jade.Boot -nomtp
 TestAgent:dtp.jade.test.TestAgent(" + mainPath +"/configuration.xml)
 InfoAgent:dtp.jade.info.InfoAgent
 DistributorAgent:dtp.jade.distributor.DistributorAgent
 CrisisManagerAgent:dtp.jade.crisismanager.CrisisManagerAgent";
\end{lstlisting}

Niestety program uruchamiany w ten sposób - który jest sposobem poprawnym, gdyż powoduje wyświetlenie okna spisu e-unitów oraz wykresu - nie generuje w efekcie żadnych wyników. Pliki które powinny pojawić się w katalogu wyjściowym nie są wogóle tworzone. Problem z całą pewnością leży w sposobie uruchamiania, ponieważ ten sam plik configuration.xml uruchomiony przy pomocy oryginalnego GUI powoduje generowanie wyników. W celu wyjaśnienia tej sytuacji skontaktowaliśmy się z autorami oprogramowania.