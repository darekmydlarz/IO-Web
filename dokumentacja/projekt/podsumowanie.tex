\chapter{Podsumowanie}

W pracy nad interfejsem webowym dla systemu planowania transportu Dispatch Rider odnieśliśmy częściowy sukces. Udało nam się spełnić tylko część spośród założonych na początku planowania projektu wymagań funkcjonalnych. 
Naszym celem było stworzenie lekkiego, możliwie niezależnego od głównego systemu Dispatch Rider, aplikacji klienckiej służącej do uruchamiania DR oraz prezentowania wyników jego pracy. Decyzja o budowie nowej aplikacji klienckiej zamiast rozwijania dotychczas istniejącej była motywowana problemami z uruchomieniem owej już istniejącej wersji webowego gui. Jak wspominaliśmy na początku dokumentacji, nie uruchamiała się ona na część systemów operacyjnych/przeglądarek, a po uruchomieniu i tak nie była w stanie komunikować się z DR.
W chwili obecnej, bogatsi o doświadczenia z pracy nad aplikacją, możem stwierdzić iż wina leżała nie tylko po stronie aplikacji webowej, ale i samego Dispatch Ridera.
Program nie posiada wyspecyfikowanych interfejsów umożliwiających komunikację z systemem webowym. W związku z tym, postanowiliśmy operować na systemie plików, w odpowiedni sposób operując danymi wejściowymi oraz wyjściowymi.
Niestety, nie udało się nam uruchomić z powodzeniem DR z linii poleceń systemu - co jest warunkiem koniecznym w każdej architekturze klient-serwer. Mimo wielokrotnych prób, eksperymentów nie udało nam się wypracować sposobu uruchomienia. Co więcej, sam autor systemu nie był w stanie poinformować nas o poprawnym sposobie uruchamiania systemu. 

Mimo powyższych problemów - które niestety uniemożliwiają nam zaprezentowanie wszystkich funkcjonalności naszej aplikacji - udało nam się spełnić pozostałe zdefiniowane przez nas i zaakceptowane przez prowadzących wymagania funkcjonalne. Aplikacja jest w stanie przesyłać oraz wizualizować dane, i może stać się w pełni użyteczna pod warunkiem naprawienia przez autora systemu DR zasygnalizowanych powyżej błędów.
\\\\\\
Analiza spełnienia wymagań funkcjonalnych:
\\\\

\begin{tabularx}{\textwidth}{ |X|p{2cm}| }
  \hline
  Generowanie pliku określającego zlecenia, o składni: ilość-pojazdów ładowność
prędkość
 & TAK/NIE  \\
  \hline
  Generowanie pliku określającego czas nadchodzenia zleceń, o składni: nr-lini czas-zlecenia  & TAK/NIE  \\
  \hline
  Generowanie pliku zawierającego liczbę kierowców & TAK/NIE \\
  \hline
  Generowanie pliku określającego parametry ciągników, o składni: moc niezawod-
ność wygoda zużycie-paliwa typ-zaczepu & TAK/NIE \\
  \hline
  Generowanie pliku opisującego parametry naczep o składni: masa pojemność
typ-ładunku uniwersalność typ-zaczepu & TAK/NIE \\
  \hline
  Generowanie pliku opisującego parametry holonów o składni: initialCapacity =
pojemność-pojazdu mode = tryb bases = ilość-baz eUnitsCount = ilość-jednostek & TAK/NIE \\
  \hline
  Generowanie pliku konfiguracyjnego configuration.xml & TAK/NIE \\
  \hline
  Kolejkowanie zadań zlecanych systemowi & TAK/NIE \\
  \hline
  Uruchamianie systemu Dispatch Rider z użyciem dostarczonych z zewnątrz lub wyge-
nerowanych plików & TAK/NIE \\
  \hline
  Wizualizacja grafu sieci transportowej & TAK/NIE \\
  \hline
  Wizualizacja tras pokonanych przez pojazdy & TAK/NIE \\
  \hline
  Wyświetlanie informacji o każdym z holonów & TAK/NIE \\
  \hline
  Wyświetlanie zbiorczego podsumowania obliczeń zawierającego koszt, dystans, czas, parametry holonów & TAK/NIE \\
  \hline
  Wyświetlanie zbiorczego podsumowania obliczeń zawierającego koszt, dystans, czas,
parametry holonów & TAK/NIE \\
  \hline 
  Przechowywanie wyników obliczeń, w celu zaprezentowania na żądanie & TAK/NIE \\
  \hline
\end{tabularx}