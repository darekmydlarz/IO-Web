\lstset{language=Java}
\chapter{Oprogramowanie schedulera}
W celu obsługi kolejnych zadań przesyłanych do systemu, powstał szkielet oprogramowania zarządzającego kolejkowaniem oraz uruchamianiem Dispatch Ridera z kolejnymi zestawami danych.
\\
Skrypt przeszukuje podany w parametrach startowych katalog systemu plików, poszukując nowo dodanych zestawów danych. W przypadku gdy taki zestaw się pojawia, umieszcza plik 'configuration.xml' zestawu w kolejce zadań do wykonania.
\\
Następnie scheduler sprawdza czy pojawiły się pliki wynikowe działania programu, co jest jednoznaczne z zakończeniem wykonywania danego zadania. Jeśli pliki znajdują się w katalogu wynikowym, wówczas uruchamiane jest kolejne zadanie z kolejki. 
\\
Kwestią istotną jest fakt, iż aby system zadziałał, plik \texttt{configuration.xml} musi znajdować się w katalogu głównym Dispatch Ridera. Ów wymóg jest podyktowany zapewne założeniem projektowym lub błędem DR. W związku z tym, plik ten dla każdego nowo uruchamianego zadania jest kopiowany do głównego katalogu przed uruchomienim procesu. 
\\
Kwestią pozostałą do uzgodnienia pozostaje w tej chwili struktura katalogów tworzona podczas ładowania plików z danymi do systemu - scheduler działa poprawnie, jednak konieczne jest wypracowanie polityki zarządzania danymi wejściowymi. Dyskusję dotyczącą tego zagadnienia przeprowadzimy podczas tworzenia systemu wczytywania i parsowania danych wejściowych.
\\\\
\subsubsection{Dodane 25.05.2012: Problem z uzyskaniem wyników działania programu}

Jak pisaliśmy powyżej, do uruchomienia Dispatch Ridera wykorzystujemy skrypt napisany w języku Java. Wykonywane jest następujące polecenie systemowe:

\begin{verbatim}
command = "java -cp " + mainPath + "/jar/DTP.jar jade.Boot -nomtp
 TestAgent:dtp.jade.test.TestAgent(" + mainPath +"/configuration.xml)
 InfoAgent:dtp.jade.info.InfoAgent
 DistributorAgent:dtp.jade.distributor.DistributorAgent
 CrisisManagerAgent:dtp.jade.crisismanager.CrisisManagerAgent";
\end{verbatim}

Niestety program uruchamiany w ten sposób - który zgodnie z kodem wcześniej powstałego GUI jest sposobem poprawnym, nie generuje w efekcie żadnych wyników. Pliki które powinny pojawić się w katalogu wyjściowym nie są wogóle tworzone. Problem z całą pewnością leży po stronie oprogramowania DispatchRidera, a konkretnie w sposobie uruchomienia za pomocą linii poleceń, ponieważ ten sam plik configuration.xml uruchomiony przy pomocy frameworka JADE powoduje generowanie wyników. W celu wyjaśnienia tej sytuacji skontaktowaliśmy się z autorami oprogramowania.

\subsubsection{Dodane 14.06.2012: Problemy z uruchomieniem DR przy pomocy Schedulera - ciąg dalszy}

W celu rozwiązania problemów z uruchamianiem DR nawiązaliśmy kontakt z autorem oprogramowania - Sebastianem Pisarskim. Niestety, nie był on w stanie podać rozwiązania naszego problemu ani sposobu uruchamiania DR z linii poleceń systemu (z podaniem ścieżki do pliku configuration.xml). Co więcej, w chwili obecnej nie jesteśmy nawet w stanie uruchomić z linii poleceń DR, mimo iż we wcześniejszej wersji ( z marca br. ) w losowo wybranych momentach aplikacja odpalana przy pomocy Schedulera zaczynała działać.

Podsumowując prace nad oprogramowaniem Schedulera - mimo poprawności stworzonego przez nas kodu, obsługującego operacje na systemie plików, nie udało nam się uruchomić tej funkcjonalności. Przeszkodą nie do pokonania okazało się w tym przypadku uruchomienie oprogramowania DR z linii komend - sposób wykorzystywany przez autorów wcześniej powstałego GUI webowego nie działa, a sam autor Dispatch Ridera nie był w stanie poinformować nas jaki jest poprawny sposób uruchomienia aplikacji. Również mimo podjętych wielokrotnie prób użycia różnorakich parametrów, nie udało nam się doświadczalnie uruchomić programu. Uznajemy że wykorzystaliśmy wszelkie możliwości na poznanie sposobu uruchomienia programu, i bez wypracowania polityki uruchomienia przez autora DR, scheduler nie będzie mógł zadziałać. Problem uruchomienia systemu nie leży po stronie naszego webowego klienta, lecz niestety po stronie samego Dispatch Ridera.