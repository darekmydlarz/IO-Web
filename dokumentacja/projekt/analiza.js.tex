\newcolumntype{R}{>{\raggedright\arraybackslash}X}
\chapter{Porównanie bibliotek graficznych w języku JavaScript}
Poniżej prezentujemy porównanie wybranych z kilkunastu różnych narzędzi JavaScriptowych, tych najciekawszych którymi można przedstawiać grafy.

\section{Informacje podstawowe}
\subsection{JavaScript InfoVis Toolkit}

\begin{table}[!htbp]
\begin{tabularx}{\textwidth}{ ||l|R|| }
\hline
\textbf{Adres} & \url{http://thejit.org/} \\
\hline
\textbf{Licencja} & new BSD license - możliwe kopiowanie, modyfikacja, rozpowszechnianie, sprzedaż z zachowaniem informacji o autorze; więcej: \url{https://github.com/philogb/jit/blob/master/LICENSE} \\
\hline

\textbf{Rozszerzalność} & Pod względem prawnym -- rozszerzalność możliwa. Pod względem technicznym -- kilka tysięcy linii kodu, jednakże brak sformalizowanej dokumentacji -- polega ona na komentarzach w kodzie \\
\hline

\textbf{Inne uwagi} & Bardzo ładna, używana m.in. na stronach Mozilli oraz prezydenta USA. Dokumentacja korzystania z biblioteki: \url{http://thejit.org/static/v20/Docs/index/General.html}.
Źródła: \url{https://github.com/philogb/jit} \\
\hline

\textbf{Ocena} & Jedynym problemem może być tylko maksymalna liczba generowanych węzłów. Poza tym posiada raczej wszystko, czego nam potrzeba, a w dodatku jest bardzo ładna. \\
\hline
\end{tabularx}
\caption{JavaScript InfoVis Toolkit - Informacje Podstawowe}
\end{table}


\vfill
\subsection{JSViz}

\begin{table}[H]
\begin{tabularx}{\textwidth}{ ||l|R|| }
\hline
\textbf{Adres} & \url{http://code.google.com/p/jsviz/} \\
\hline
\textbf{Licencja} & Apache 2.0 - dopuszcza użycie kodu źródłowego zarówno na potrzeby wolnego oprogramowania, jak i zamkniętego oprogramowania komercyjnego. \\
\hline

\textbf{Rozszerzalność} & Licencja zezwala, aczkolwiek brak dokumentacji, jedynie komentarze w kodzie o długości kilku/kilkunastu tysięcy linii. \\
\hline

\textbf{Inne uwagi} & Niedostatecznie przyjemna dla oka. \\
\hline

\textbf{Ocena} & Brak jakichkolwiek etykiet, prawie żadna możliwość wpływu na wygląd generowanego grafu, dość długie i chaotyczne generowanie się grafu, biblioteka z przed 5 lat, nie zachęca wyglądem. \\
\hline
\end{tabularx}
\caption{JSViz - Informacje Podstawowe}
\end{table}


\vfill
\subsection{Dracula}

\begin{table}[H]
\begin{tabularx}{\textwidth}{ ||l|R|| }
\hline
\textbf{Adres} & \url{http://www.graphdracula.net/} \\
\hline
\textbf{Licencja} & MIT (X11) license - możliwe są kopiowanie, modyfikacja, rozpowszechnianie, w tym sprzedaż. \\
\hline

\textbf{Rozszerzalność} & Pod względem prawnym -- rozszerzalność możliwa. Pod względem technicznym -- ok. 10 tysięcy linii kodu, jednakże brak sformalizowanej dokumentacji -- polega ona na komentarzach w kodzie i udostępnieniu pojedynczego przykładu na stronie projektu oraz skomentowanych przykładów możliwych do ściągnięcia. \\
\hline

\textbf{Inne uwagi} & Niewystarczająco wydajna, brak wbudowanych mechanizmów ,,ładnego'' rozkładania grafów w przestrzeni. \\
\hline

\end{tabularx}
\caption{Dracula - Informacje Podstawowe}
\end{table}




\vfill
\newpage
\section{Funkcjonalność}
\subsection{JavaScript InfoVis Toolkit}

\begin{table}[H]
\begin{tabularx}{\textwidth}{ ||l|R|| }
\hline
\textbf{Interaktywne węzły} & Tak \\
\hline
\textbf{Interaktywne ścieżki} & Nie \\
\hline
\textbf{Zoom} & Tak \\
\hline
\textbf{Opis węzłów} & Tak \\
\hline
\textbf{Opis ścieżek} & Tak \\
\hline
\textbf{Automatyczna zmiana kształtu} & Nie \\
\hline
\textbf{Inne uwagi} & Możliwość przesuwania grafu. Wczytuje dane w formacie JSON \\
\hline

\end{tabularx}
\caption{JavaScript InfoVis Toolkit - Funkcjonalność}
\end{table}


\subsection{JSViz}

\begin{table}[H]
\begin{tabularx}{\textwidth}{ ||l|R|| }
\hline
\textbf{Interaktywne węzły} & Tak \\
\hline
\textbf{Interaktywne ścieżki} & Nie \\
\hline
\textbf{Zoom} & Nie \\
\hline
\textbf{Opis węzłów} & Nie \\
\hline
\textbf{Opis ścieżek} & Nie \\
\hline
\textbf{Automatyczna zmiana kształtu} & Nie \\
\hline
\textbf{Inne uwagi} & Wczytuje dane w formacie XML\\
\hline

\end{tabularx}
\caption{JSViz - Funkcjonalność}
\end{table}


\subsection{Dracula}

\begin{table}[H]
\begin{tabularx}{\textwidth}{ ||l|R|| }
\hline
\textbf{Interaktywne węzły} & Tak \\
\hline
\textbf{Interaktywne ścieżki} & Nie \\
\hline
\textbf{Zoom} & Nie \\
\hline
\textbf{Opis węzłów} & Tak \\
\hline
\textbf{Opis ścieżek} & Tak \\
\hline
\textbf{Automatyczna zmiana kształtu} & Nie \\
\hline
\textbf{Inne uwagi} & W bibliotece zaimplementowane takie algorytmy jak: Bellman-Ford, Dijkstra, Floyd-Warshall, quicksort, selectionsort, mergesort, topologicalsort. W kodzie wiele funkcji czekających na implementacje.\\
\hline
\end{tabularx}
\caption{Dracula - Funkcjonalność}
\end{table}


\section{Wydajność}
Dla wyżej wymienionych bibliotek przeprowadziliśmy testy wydajnościowe. Ich wyniki są następujące:

\subsection{JavaScript InfoVis Toolkit}

\begin{table}[H]
\begin{tabularx}{\textwidth}{ ||l|R|R|| }
\hline
\textbf{Węzłów} & \textbf{Brak krawędzi} & \textbf{Krawędzie} \\
 &  & 0-3 dla każdego węzła \\
\hline
\textbf{100} & Generuję się od razu, można bez problemu korzystać. & Generuję się kilka sekund, korzysta się wygodnie. \\
\hline
\textbf{200} & Powoduje kilku/kilkunastosekundowe generowanie się grafu. & Podobnie, dodatkowo utrudnia wygodne korzystanie z grafu.	 \\
\hline
\textbf{500} & Powoduje kilkunastosekundowe generowanie się, korzystanie z grafu raczej niemożliwe. & Generowanie trwa kilka minut (zdecydowanie za długo).	 \\
\hline
\textbf{1000} & Generowanie trwa kilka minut. &  \\
\hline
\end{tabularx}
\caption{JavaScript InfoVis Toolkit -- Wydajność}
\end{table}

\subsection{JSViz}

\begin{table}[H]
\begin{tabularx}{\textwidth}{ ||l|R|| }
\hline
\textbf{Węzłów} & \textbf{Krawędzie/brak krawędzi (0-3 dla każdego węzła)} \\
\hline
\textbf{100} & Generowanie trwa kilkadziesiąt sekund. \\
\hline
\textbf{200} & Generowanie trwa kilkadziesiąt sekund.	 \\
\hline
\textbf{500} & Generuje się kilka minut, nie można wygodnie korzystać.  \\
\hline
\end{tabularx}
\caption{JSViz -- Wydajność}
\end{table}

\subsection{Dracula}

\begin{table}[H]
\begin{tabularx}{\textwidth}{ ||l|R|| }
\hline
\textbf{Węzłów} & \textbf{Krawędzie/brak krawędzi (0-3 dla każdego węzła)} \\
\hline
\textbf{500} & Powoduje zauważalny, ok 3 sek. narzut czasowy na tworzenie grafu.  \\
\hline
\textbf{1000} & Zawiesza przeglądarkę. \\
\hline
\end{tabularx}
\caption{Dracula -- Wydajność}
\end{table}

\vfill
\newpage
\section{Wnioski}
Zgodnie z powyższym porównaniem różnych bibliotek podjęliśmy decyzję, iż w naszym projekcie skorzystamy z możliwości jakie daje nam biblioteka JavaScriptInfoVis Toolkit. Jest ona najmłodszym z pośród przeglądanych projektów, ale jednocześnie jest bardzo rozbudowana i dopracowana. Pozwala na rysowanie wielu rodzajów grafów i robi to w bardzo ładny dla oka sposób. Dobrą rekomendacją dla niej jest także fakt, iż wykorzystywana jest przez Fundację Mozilla na swoich stronach, a także można ją znaleźć na oficjalnej stronie prezydenta Stanów Zjednoczonych. Kod biblioteki został napisany w dość czytelny sposób, korzystanie z niej nie powinno stanowić większych problemów. Licencja, na której jest udostępniana pozwala na wykorzystanie jej w naszym projekcie.