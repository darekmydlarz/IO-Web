\chapter{Sformułowanie zadania projektowego}
Przedmiotem naszego przedsięwzięcia będzie zbudowanie czytelnego 
i prostego w obsłudze webowego GUI do istniejącego już systemu
\emph{Dispatch Rider} zajmującego się problemem transportowym.
Chcielibyśmy stworzyć narzędzie na tyle wygodne, by korzystanie z
niego nie stanowiło problemu dla nowych jego użytkowników.

\section{Zapoznanie się z dotychczasowym systemem}
\subsection{Ustalenia}
\begin{enumerate}
\item Otrzymany projekt nie zachowuje się deterministycznie pod różnymi platformami. Pod Linuxem (Ubuntu) uruchomione zadanie nie oblicza się, natomiast pod Windowsem (7) GUI uruchamia się niedeterministycznie. Nie jesteśmy w stanie określić co powoduje owe problemy, jednakże mogą one uniemożliwić rozszerzenie funkcjonalności GUI. Faktem istotnym jest również to, że aplet zamiast uruchamiać się w przeglądarce, raportuje błąd i nie uruchamia się.
\item W czasie analizy szczególną uwagę przyłożyliśmy do opisu problemów z którymi spotkały się zespoły wcześniej realizujące projekty. Z ich uwag wynika iż DispatchRider posiada błędy, które w szczególności mogą skutkować brakiem pliku wynikowego, nieuwzględnieniem sieci transportowej, nieuwzględnieniem czasu utworzenia holonu.
\item Dane wykorzystywane do wizualizacji wyników oraz tworzenia statystyk, są przekazywane za pośrednictwem pliku wyjściowego generowanego przez DR. Otwiera to możliwość zmiany koncepcji tworzenia GUI, ponieważ moduł nie jest w bezpośredni sposób powiązany z kodem DispatchRidera. Jednocześnie powoduje to konieczność przemyślanego zaprojektowania metody przekazywania owych danych do GUI.
\end{enumerate}

\subsection{Wnioski}
\begin{enumerate}
\item W świetle wyżej zarysowanych problemów, chcielibyśmy zaproponować pewną zmianę koncepcji tworzenia graficznego interfejsu użytkownika.
\item Chcielibyśmy rozwiązać problemy z uruchamieniem GUI na różnych systemach. Mimo iż aplety Java powinny uruchamiać się bezproblemowo, empirycznie sprawdziliśmy że GUI nie działa prawidłowo na poszczególnych systemach operacyjnych. Aby zapobiec tym problemom, chcielibyśmy stworzyć nową aplikację webową i wykorzystać przy tym takie technologie jak: HTML5, CSS, JavaScript. Z naszego doświadczenia wynika iż dobrze zaprojektowane z użyciem w.w. GUI działać będzie na każdym systemie i każdej przeglądarce. Powodem dla którego preferujemy akurat te technologie, a nie aplety Javy, jest ich nowoczesność - pragniemy wykorzystać HTML5, zapewniający olbrzymie możliwości, choćby w dziedzinie modelowania i prezentacji grafów. Z przykrością musimy stwierdzić, iż dotychczasowe sposoby wizualizacji grafów w interfejsie DispatchRidera charakteryzowały się pewnym brakiem czytelności, oraz interaktywności.
\item Budowana przez nas aplikacja podobnie jak dotychczasowe projekty GUI, korzystałaby z plików wyjściowych DispatchRidera. Oddzielenie interfejsu użytkownika od głównego ciała programu pozwala na komfortową pracę, bez konieczności zagłębiania się w kod głównego programu i jego modyfikacji, co mogłoby nawet uniemożliwić jakiekolwiek widoczne postępy w rozwijaniu przez nas dotychczasowego GUI.
\item Projektując GUI z użyciem naszych technologii, chcielibyśmy odtworzyć zaimplementowane w wersji z apletami Javy funkcjonalności, mając jednak stuprocentową pewność że w przeciwieństwie do niej, nasz interfejs będzie w pełni użytkowalny, i będzie możliwe jego uruchomienie na każdym systemie i przeglądarce obsługującej HTML5.
\item Chcielibyśmy skupić się zwłaszcza na problemie prezentacji tras, zwracając dużą uwagę na ergonomię działania programu. Użycie w.w. technologii umożliwia tworzenie interaktywnych prezentacji grafu na które kładziemy dużą uwagę, oraz umożliwia właściwie nieograniczone możliwości modyfikowania ich, zmiany perspektywy oraz ilości przekazywanych informacji. W chwili obecnej, interfejs graficzny jest wyjątkowo nieergonomiczny, co wynika choćby z faktu iż jego uruchomienie jest czynnością niedeterministyczną. Mamy nadzieję że nasz pomysł warstwy prezentacji spowoduje jakościową zmianę w sposobie obsługi systemu.
\item Naszym zdaniem, kontynuacja istniejących implementacji graficznego interfejsu może przerodzić się w klasyczny marszu ku klęsce. Aby temu zapobiec, postulujemy zmianę koncepcji tworzenia GUI i zbudowanie podwalin dobrze zaimplementowanego, udokumentowanego, łatwego w dalszym rozwoju, a przede wszystkim działającego GUI.
\end{enumerate}
